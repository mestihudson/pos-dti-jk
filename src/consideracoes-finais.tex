\section{Considerações finais}

  A despeito de apresentar uma aplicação de maneira completa enquanto solução para o problema proposto, fica evidente a sua capacidade em demonstrar a sustentabilidade do método utilizado para a construção daquela, vez que os mecanismos de proteção contra processos regressivos são efetivos e criam uma rede de segurança que dá confiança a equipe envolvida em seguir os objetivos traçados sem o receio de não ter antecipado riscos para os quais só se tenha ciência em momento ulterior do projeto.

  Assim, cada um dos componentes dessa abordagem cumpre papéis fundamentais nessa jornada: desenvolvimento guiado por testes define a linha conceitual, centrada na linguagem de negócios alvo; refatoração permite alterar as estruturas internas da solução sem tocar nas especificações comportamentais estabelecidas antes, inclusive, permitindo, responsavelmente, procrastinar o desenho arquitetural do código para momento adequado; e, por fim, arquitetura limpa, que, dependendo da maneira que se escolha desenvolver, já pode iniciar por ela (não foi o nosso caso, por questão puramente didática), capacita o código a ser extendido e mantido a medida que novas demandas são apresentadas, sem o engessamento que inviabiliza a sua continuidade.
