\begin{singlespace}
  \resumo
    {Abstract}
    {
      It is not an easy task to conceptualize TDD, since some authors put it as a technique while others as a philosophy and, in the end, it is a little of both and perhaps more, which, due to the poverty of human language, it is not possible to express. The fact is that he awakens very different feelings, while some love him like a god, others hate him like a devil, and this is due to the fact that it is no simple task to dominate him.
      Clean Architecture is a term that is in fashion, although its concepts are well rehashed from other authors who preceded it, and this makes it even more robust, as it has been put to the test and has passed with flying colors.
      But how do we go from a shapeless mass of code where concepts are not well matured to an example of a well-finished product? Bringing together what these two concepts have the best to offer.
      This paper aims to present a method to use test-driven development to get a clean architecture in cutting of building a multiple users web game application. A user story is selected among a lot of them raised to prove the approach.
    }
    {Keywords}
    {test-driven development, refactoring, clean architecture}

  \resumo
    {Resumo}
    {
      Não é tarefa fácil conceituar TDD, posto que alguns autores o colocam como técnica enquanto outros como uma filosofia e, no final, ele é um pouco de ambos e talvez mais, que por pobreza da linguagem humana não seja possível expressar. Fato é que ele desperta sentimentos bem diferentes, enquantos unso amam como a um deus outros os odeiam como a um demônio, e isso se deve ao fato de não é tarefa simples dominá-lo.
      Arquitetura Limpa, é um termo que está na moda, ainda que seus conceitos sejam bem requentados de outros autores que o precederam, e isso a torna ainda mais robusta, vez que foi posta sob prova e tem passado com louvor.
      Mas como sair de uma massa disforme de código onde os conceitos não estão bem amadurecidos para um exemplo de produto bem acabado? Unindo o que esses dois conceitos tem de melhor a oferecer.
      Este trabalho visa apresentar um método para o uso de desenvolvimento guiado por testes para obter uma arquitetura limpa em um recorte da construção de uma aplicação de jogo com múltiplos competidos para internet. Uma história de usuário é selecionada dentre várias levantadas para demonstrar a abordagem.
    }
    {Palavras-chave}
    {desenvolvimento guiado por testes, refatoração, arquitetura limpa}
\end{singlespace}
