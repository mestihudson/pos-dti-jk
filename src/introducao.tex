\section{Introdução}

  Desenvolver software sempre foi uma atividadade complexa, que pode, facilmente,  apresentar desafios das mais diveras naturezas: agendamento; má-definição de  requisitos - funcionais ou não; sub - ou super - dimensionamento de recursos; não-escalabilidade; etc.

  Negligenciar esse amálgama de possibilidades é um erro comum, e até esperado, de profissionais pouco experimentados, mas, não raro, encontram-se exemplares com bastante quilometragem de carreira a cometê-los.

  O Standish Group demostra, em seu \emph{Chaos Report}\cite{ChaosReport2015}, que quanto maior o projeto maior é a chance de falha. O mesmo relatório, no entanto, aponta alguns caminhos a seguir para encontrar sucesso, dentre eles: \emph{um processo ágil} e \emph{uma arquitetura padrão}.

  Ao mencionar agilidade é possível inferir que a capacidade de adaptação é uma  necessidade fundamental para o bom andamento, princípio defendido desde a  primeira hora por este tipo de metodologia \cite{ManifestoAgil2001}.

  Quanto à arquitetura, faz-se evidente que, utilizar um padrão é uma decisão que pode poupar longas, e pouco frutíferas, discursões sobre assuntos já consolidados \emph{de facto} ou na literatura acadêmica.

  Em meio as publicações especializadas é possível encontrar trabalhos relacionando a melhora da qualidade e da produtividade no desenvolvimento de software com o uso de testes para guiar tal atividade - \cite{Moggi2011}, \cite{Filho2012} e \cite{Benato2021}. Outro salienta os vínculos da chamada \emph{gestão de dívida técnica} com uma \emph{arquitetura limpa} - \cite{Beltrao2020} - e, ainda, à \emph{flexibilidade evolutiva do software} é destacada por \cite{Souza2021}. Nenhum deles, no entanto, traça um caminho de uma ponta a outra, unindo os dois conceitos. Esta é a tarefa do presente trabalho ou, objetivamente, \emph{realizar um recorte em um projeto de software para demonstrar o uso de desenvolvimento guiado por testes no suporte para alcançar uma arquitetura limpa}.

  De maneira a tornar a apresentação fluída, a seguinte estrutura foi estabelecida. A seção 2 apresenta os conceitos básicos. A seção 3 discute os trabalhos relacionados. A seção 4 expõe o problema como um todo e distingue o recorte que será utilizado na demonstração pretendida. A seção 5 detalha a proposta de solução intencionada. A seção 6 conlui o trabalho.
