\section{Introdução}

  Desenvolver software sempre foi uma atividadade complexa, que pode, facilmente,  apresentar desafios das mais diveras naturezas: agendamento; má-definição de  requisitos - funcionais ou não; sub - ou super - dimensionamento de recursos; não-escalabilidade; etc.

  Negligenciar esse amálgama de possibilidades é um erro comum, e até esperado, de profissionais pouco experimentados, mas, não raro, encontram-se exemplares com bastante quilometragem de carreira a cometê-los.

  O Standish Group demostra, em seu \emph{Chaos Report}\cite{ChaosReport2015}, que quanto maior o projeto maior é a chance de falha. O mesmo relatório, no entanto, aponta alguns caminhos a seguir para encontrar sucesso, dentre eles: \emph{um processo ágil} e \emph{uma arquitetura padrão}.

  Ao mencionar agilidade é possível inferir que a capacidade de adaptação é uma  necessidade fundamental para o bom andamento, princípio defendido desde a  primeira hora por este tipo de metodologia \cite{ManifestoAgil2001}.

  Quanto à arquitetura, faz-se evidente que, utilizar um padrão é uma decisão que pode poupar longas, e pouco frutíferas, discursões sobre assuntos já consolidados \emph{de facto} ou na literatura acadêmica.
