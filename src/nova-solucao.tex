\subsection{A nova solução}

  Em função das limitações de tamanho e tempo disponível para conclusão,
  observa-se uma boa oportunidade para utilização de uma abordagem baseada em
  metodologias ágeis. No entanto, como não é objetivo discorrer sobre o
  andamento de um projeto de software em sua completude mas focar nas técnicas
  de desenvolvimento utilizadas, que podem ser aplicadas, inclusive, no contexto
  de mecanismos tradicionais, uma certa liberdade é tomada.

  Usando histórias de usuário como método de requisitos observamos a existência
  de muitas delas, de tal forma que se faz necessário organizar as mesmas dentro
  de um fluxo de capacidade mínimo\footnotemark para completar as intenções
  firmadas no escopo. Daí se chega à \textbf{Figura
  \ref{fig:mapeamento-historias-usuarios}}, onde se observa que 6 hisórias são essenciais: criar
  conta de usuário; realizar autenticação; chamar partida; atender chamado para
  partida; colocar peça; e, movimentar peça. Também é fácil perceber, que movimentar peça
  ainda possui um nível de abstração muito alto\footnote{\cite[pág. 23]
  {Cohn2004}}, sugerindo uma divisão da mesma para alcançar a \textbf{Figura
  \ref{fig:divisao-historia-executar-partida}}.
  \footnotetext{\cite{Patton2014} descreve apresenta uma técnica de
  mapeamente bastante útil, em particular para o contexto em que não existe um
  proprietário do projeto capaz de priorizar as histórias que mais lhe agregam
  valor.}

  Factualmente, existem duas escolas de TDD: Londres e Chicago, também conhecidas
  como \emph{de fora para dentro} (Outside in) e \emph{de dentro para fora}
  (Inside out), ou ainda, Mockista e Clássica. Segundo \citeonline[págs. 488-492]
  {BryantPerez2019} sua tecnologia, mecanismos de construção, habilidades e até
  preferências pessoais irão determinar que abordagem escolher. Mas também cita
  algumas características que ajudam nessa, tais como: já possui bem delineadas
  as regras do negócio ou terá que traduzir o linguajar dos usuários e decodificar
  o processo a ser automatizado; existem restrições de plataforma impostos (se é
  uma aplicação web, se deve usar um determinado servidor de banco de dados, etc).
  Advogo ainda que é possível utilizar ambas (claro, em diferentes níveis).

  Sendo assim, optamos pela escola de Londres, que tem como principais
  representantes \citeonline{FreemanPryce2009}. Sob esta há que se observar que:
  o clico de TDD possui dois níveis de execução (\textbf{Figura \ref{fig:ciclo-tdd-outside-in}}).
  Graças a isso, é esperado que o ciclo mais externo permanesça por mais tempo
  no estado de falha. Também somos forçados a definir um esboço arquitetural
  sobre o qual repousaremos a solução inicialmente (que pode, ou não, vir a ser
  modificado durante o transcorrer do projeto).
