\subsection{A solução}

  Em função das limitações de tamanho e tempo, observou-se uma oportunidade para utilização de abordagem baseada em metodologias ágeis. No entanto, não é objetivo discorrer sobre o andamento de um projeto de software em sua completude mas, focar em algumas técnicas de desenvolvimento utilizadas. De tal forma, contamos com uma certa indulgência do leitor no que se refere a omissões descritivas dos caminhos gerenciais tomados.

  Usando \emph{histórias de usuário} como método de requisitos observamos a existência de muitas delas, de tal forma que se faz necessário organizar as mesmas dentro de um fluxo de capacidade mínimo \footnote{\citeonline{Patton2014} apresentam uma técnica, \emph{Mapeamento de Histórias de Usuário}, bastante útil para identificar fluxos e definir sua importância, em particular para o contexto em que não existe um proprietário de projeto - PO - capaz de priorizar as histórias que mais lhe agregam valor.} para completar as intenções firmadas no escopo. Daí se chega à \figref{mapeamento-historias-usuario}, onde se observa que 7 histórias são essenciais: \emph{criar conta; entrar; sair; chamar partida; atender chamado para partida; colocar peça em jogo; e, movimentar peça}. Sentimos, por experiência, que a última tem tendência para um \emph{épico}\footnote{\cite[pág. 6]{Cohn2004}}, por si só, mas,  para evitar a \emph{paralisia de análise}\footnote{\cite[pág. 71]{Pugh2011}}, vamos procrastinar a decisão de dividí-la\footnote{\cite[pág. 24]{Cohn2004}}, até que lidemos com ela especificamente.

  \imagem
    {width=16cm}
    {Mapeamento de histórias de usuário}
    {mapeamento-historias-de-usuario.miro}
    {mapeamento-historias-usuario}
    {Próprio autor\footnotemark}
  \footnotetext{Utilizando o serviço https://miro.com/}

  Factualmente, existem duas escolas de TDD: \emph{Chicago} - encontrada, ainda, como inside-out (de dentro para fora), classicista, bottom-up (de baixo para cima) - e \emph{Londres} - também conhecida como outside-in (de fora para dentro), mockista, top-down (de cima para baixo). Segundo \citeonline[págs. 488-492] {BryantPerez2019}, a tecnologia utilizada, mecanismos de construção, habilidades e preferências pessoais irão determinar que abordagem escolher. Mas também cita algumas características que ajudam nesta: possui bem delineadas as regras do negócio ou terá que traduzir o processo dos interessados em regras que possam ser automatizadas; existem restrições de plataforma (deve ser uma aplicação web, deve usar um determinado servidor de banco de dados, etc).

  Sendo assim, optamos pela abordagem londrina, que tem como principais representantes \citeonline{FreemanPryce2009}. A primeira observação é sobre o ciclo que, nessa perspectiva, possui dois níveis de execução (\figref{ciclo-atdd}). Graças a isso, é esperado que a etapa mais externa permanesça por mais tempo em estado de falha.

  \imagem
    {width=6cm}
    {Duplo ciclo da escola londrina de tdd}
    {ciclo-atdd}
    {ciclo-atdd}
    {Próprio autor\footnotemark}
  \footnotetext{Baseado na imagem encontrada em \citeonline[pág. 40]
  {FreemanPryce2009}}

  Também somos forçados a definir um esboço estrutural dos componentes (\figref{esboco-estrutural}) de auto nível sobre o qual repousaremos a solução (inicialmente - pois ela pode, ou não, vir a ser modificado durante o transcorrer do projeto).

  \imagem
    {width=6cm}
    {Esboço estrutural dos componentes da solução}
    {esboco-estrutural}
    {esboco-estrutural}
    {Próprio autor}

  \subsubsection{Criar conta}

    Para levantar as histórias, \citeonline{Cohn2004} sugere o seguinte método: identificar os papéis; utilizá-los para construir narrativas no formato - \narrativa{papel}{fazer alguma coisa}{atingir um objetivo}; levantar os critérios de aceite de cada história.

    Com a história \emph{Criar conta} chegamos a seguinte narrativa: \narrativa{visitante}{criar uma conta}{entrar e jogar uma partida}.

    Também foram levantados os seguintes critérios de aceite: deve ser informado um email válido; deve ser informada uma senha de tamanho mínimo 8 (oito) e tamanho máximo 100 (cem) caracteres; não deve ser possível criar uma conta para um email que já vinculado a uma outra conta; deve constar um registro na base de dados.

    Há elementos para elencarmos os cenários a serem trabalhados (\figref{criar-conta.cenarios}).

    \codigo
      {numbers=none}
      {\textbf{Cenários} elencados para a \textbf{história} \emph{Criar conta}}
      {criar-conta.cenarios}
      {criar-conta.cenarios}
      {Próprio autor}

    Agora escolhemos um deles para detalhamento e implementação\footnote{\citeonline{Nicieja2018} apresenta um método para derivação de cenários a partir dos critérios de aceite usando a linguagem \emph{Gherkin}, que se baseia no famoso artigo de \citeonline{North2006}.}: \emph{deve ser bem sucedido}. Daí chegamos as passos da \figref{criar-conta.deve-ser-bem-sucedido.cenario}.

    \codigo
      {numbers=none}
      {Passos do \textbf{cenário} \emph{deve ser bem sucedido} da \textbf{história} \emph{Criar conta}}
      {criar-conta.deve-ser-bem-sucedido.cenario}
      {criar-conta.deve-ser-bem-sucedido.cenario}
      {Próprio autor}

  Essa estrutura se baseia na seguinte ideia: o passo identificado por \textbf{dado} define o \emph{contexto} - pré-condição - onde a \emph{ação} (\textbf{quando}) deve acontecer para que o \emph{resultado esperado} - pós-condição - (\textbf{então}) seja observado.

  Há de salientar que os passos do cenário descrito são, suficientemente, abstratos para permitir que os não-técnicos o leiam (ou, ainda, desejavelmente, os escrevam) mas não tanto que inviabilizer a sua automatização. Construímos o que \citeonline[pág. 24]{Evans2003} chama de \emph{linguagem ubíqua}. Dentre outros benefícios se pode citar a não fragilidade\footnote{\cite[pág. 93]{RoseWynneHellesoy2015} Brittle features} desses artefatos, tornando-os resilientes às alterações de implementação, enquanto o comportamento esperado se mantiver intácto. Essa característica é fundamental para alicerçar a \emph{refatoração}\footnote{cite{Fowler1999}} e, por sua vez, é condição para o alvo mirado, a \emph{arquitetura limpa}.

  Retomando o esboço estrutural, o enriquecemos com os componentes que substituirão a figura humana no processo, dando vida própria a documentação\footnote{\citeonline[pág. 29]{Adzic2011} Live documentation}, donde se chega a \figref{esboco-estrutural-enriquecido}.

  \imagem
    {width=12cm}
    {Esboço estrutural enriquecido com os componentes que permitem a automatização}
    {esboco-estrutural-enriquecido}
    {esboco-estrutural-enriquecido}
    {Própio autor}

  Existem muitos mitos em torno das metodologias ágeis, e uma delas é a não existência de modelagem. Essa falsa crença se dá, muitas vezes, por uma noção equivocada de que documentação e modelos são equivalentes. \citeonline{Ambler2002} esses mitos e mostra que essa atividade é, além de possível, essencial para um bom projeto de software. Isso nos permite, por exemplo, rabiscar uma solução para o cenário descrito, como pode ser observado na \figref{modelo-solucao.criar-conta-deve-ser-bem-sucedido.jpeg}.

  \imagem
    {width=12cm}
    {Modelo visual para solucionar o \textbf{cenário} \emph{deve ser bem sucedido} da \textbf{história} \emph{Criar conta}}
    {modelo-solucao.criar-conta-deve-ser-bem-sucedido.jpeg}
    {modelo-solucao.criar-conta-deve-ser-bem-sucedido.jpeg}
    {Própio autor}

  Esse ponto é auxiliado pelo que \citeonline[págs. 142 e 226]{Abelson1996} chama de \emph{wishful thinking}\footnote{Pensamento positivo, em tradução livre.} - mais precisamente o que \citeonline[pág. 45]{Astels2003} denomina \emph{programar por intenção} - para implementar os passos do cenário apresentados como se a funcionalidade pretendida já existisse. O objetivo é ser o mais claro possível sobre o que se almeja com aquela especificação, o resultado pode ser visto na \figref{criar-conta.deve-ser-bem-sucedido.passos}.

  \codigo
    {numbers=none, basicstyle=\ttfamily\tiny}
    {Implementação dos passos do \textbf{cenário} \emph{deve ser bem sucedido} da \textbf{história} \emph{Criar conta}}
    {criar-conta.deve-ser-bem-sucedido.passos}
    {criar-conta.deve-ser-bem-sucedido.passos}
    {Própio autor}

  Essa etapa pode divergir bastante entre os diversos autores que tratam do tema, no que tange os conceitos de \emph{baby steps}\footnote{\cite[pág. 1]{Aniche2011}}, \emph{iniciar pelas verificações}, \emph{usar somente uma asserção por cenário}, \emph{não usar asserções fora da fase de verificação}. No fim, prevalecem, \emph{a familiaridade com as tecnologias empregadas}, \emph{a experiência no processo de desenvolvimento guiado por testes}, \emph{a simplicidade}\footnote{\cite[pág. TODO]{Beck2003}}, mas, principalmente, \emph{o bom senso}.

  Como dito antes, o modelo londrino supõe a ciência prévia de determinados componentes da solução, de tal forma que o código da especificação funciona como uma tela de proteção no picadero de um circo funciona para os artistas do trapézio. Esse, talvez, seja o motivo de, algumas vezes, se tomar atalhos como o não uso de testes unitários em cenários em que, pela experiência, não apresentam grandes desafios. Ou seja, o desenvolvedor sabe o que quer fazer, mas para evitar horas procurando uma vírgula mal colocada, ele se utiliza da espeficifação automatizada para validar o caminho que ele está tomando. Essa mesma, lhe garantirá a não regressão de um caso, no momento em que um caminho mais elegante, internamente falando, for imaginado.

  Para este cenário, em particular, vemos, claramente, 2 componentes fundamentais: entrada de dados - ui (interface de usuário, mas poderia ser uma interface de código, se nossa solução fosse destinada a outros sistemas ao invés de humanos); e, saída de dados - db (que poderia envolver outros atores, como, por exemplo, um servidor de envio de mensagens eletrônicas externo, caso fossemos exigir ativação da conta por algum link encaminhado por email).

  Esperamos que, ao final da execução deste caso de teste, durante a verificação dos \emph{resultados esperados}, tenhamos um registro que nos comprove que a entrada de dados foi devidamente persistida. Isso deve ocorrer em contraposição ao \emph{contexto} do teste, em que o dito registro não deve existir. Assim nosso primeiro passo na direção da completude do cenário é a criação da consulta que faz a verificação da ausência do registro na base de dados (\figref{criar-conta-deve-ser-bem-sucedido-passo.dado}).

  \codigo
    {numbers=none, basicstyle=\ttfamily\tiny}
    {Implementação do código de suporte do passo de contexto do cenário}
    {criar-conta-deve-ser-bem-sucedido-passo.dado}
    {criar-conta-deve-ser-bem-sucedido-passo.dado}
    {Própio autor}

  Para o passo que representa a ação, vemos que existem níveis de abstração, ao observar que o comando \emph{quando eu me registro}, na verdade é uma composição de 3 ações: abrir a página do formulário de registro; preencher os campos obrigatórios do mesmo (que, por si só, já é uma composição pois resume o preenchimento de muitos campos a uma única ação composta); e, finalmente, submeter essas informações pelo clique no botão \emph{registrar} (\figref{criar-conta-deve-ser-bem-sucedido-passo.quando}).

  \codigo
    {numbers=none, basicstyle=\ttfamily\tiny}
    {Implementação do código de suporte do passo de ação do cenário}
    {criar-conta-deve-ser-bem-sucedido-passo.quando}
    {criar-conta-deve-ser-bem-sucedido-passo.quando}
    {Própio autor}

  Por fim, temos o passo de verificação, como já havíamos adiantado, reutilizamos o código de suporte da pré-condição que deve ter o efeito oposto nesse momento, bem como a da existência de uma mensagem de feedback do sucesso da ação (\figref{criar-conta-deve-ser-bem-sucedido-passo.entao}).

  \codigo
    {numbers=none, basicstyle=\ttfamily\tiny}
    {Implementação do código de suporte do passo de verificação do cenário}
    {criar-conta-deve-ser-bem-sucedido-passo.entao}
    {criar-conta-deve-ser-bem-sucedido-passo.entao}
    {Própio autor}

  Isso conclui o trabalho de construção da rede de proteção que nos permitirá implementar a funcionalidade com a confiança de que seremos apoiados pela especificação automatizada. Desse momento em diante, podemos, inclusive, escolher a plataforma que melhor nos convier (desde que atenda aos limites mínimos impostos, que nesse caso são: uma aplicação web e um banco de dados relacional suportado pela plataforma). Ou seja, a miriade de possibilidades em que essa solução pode ser implementada é enorme. Para o nosso caso, em particular, escolhemos um frontend em React e um backend em Java/Quarkus e um banco de dados PostgreSQL.
