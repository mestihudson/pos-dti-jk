\subsection{A solução}

  Em função das limitações de tamanho e tempo, observou-se uma oportunidade para utilização de abordagem baseada em metodologias ágeis. No entanto, não é objetivo discorrer sobre o andamento de um projeto de software em sua completude mas, focar em algumas técnicas de desenvolvimento utilizadas. De tal forma, contamos com uma certa indulgência do leitor no que se refere a omissões descritivas dos caminhos gerenciais tomados.

  Usando \emph{histórias de usuário} como método de requisitos observamos a existência de muitas delas, de tal forma que se faz necessário organizar as mesmas dentro de um fluxo de capacidade mínimo \footnote{\citeonline{Patton2014} apresentam uma técnica, \emph{Mapeamento de Histórias de Usuário}, bastante útil para identificar fluxos e definir sua importância, em particular para o contexto em que não existe um proprietário de projeto - PO - capaz de priorizar as histórias que mais lhe agregam valor.} para completar as intenções firmadas no escopo. Daí se chega à \figref{mapeamento-historias-usuario}, onde se observa que 7 histórias são essenciais: \emph{criar conta; entrar; sair; chamar partida; atender chamado para partida; colocar peça em jogo; e, movimentar peça}. Sentimos, por experiência, que a última tem tendência para um \emph{épico}\footnote{\cite[pág. 6]{Cohn2004}}, por si só, mas,  para evitar a \emph{paralisia de análise}\footnote{\cite[pág. 71]{Pugh2011}}, vamos procrastinar a decisão de dividí-la\footnote{\cite[pág. 24]{Cohn2004}}, até que lidemos com ela especificamente.

  \imagem
    {width=16cm}
    {Mapeamento de histórias de usuário}
    {mapeamento-historias-de-usuario.miro}
    {mapeamento-historias-usuario}
    {Próprio autor\footnotemark}
  \footnotetext{Utilizando o serviço https://miro.com/}

  Factualmente, existem duas escolas de TDD: \emph{Chicago} - encontrada, ainda, como inside-out (de dentro para fora), classicista, bottom-up (de baixo para cima) - e \emph{Londres} - também conhecida como outside-in (de fora para dentro), mockista, top-down (de cima para baixo). Segundo \citeonline[págs. 488-492] {BryantPerez2019}, a tecnologia utilizada, mecanismos de construção, habilidades e preferências pessoais irão determinar que abordagem escolher. Mas também cita algumas características que ajudam nesta: possui bem delineadas as regras do negócio ou terá que traduzir o processo dos interessados em regras que possam ser automatizadas; existem restrições de plataforma (deve ser uma aplicação web, deve usar um determinado servidor de banco de dados, etc).

  Sendo assim, optamos pela abordagem londrina, que tem como principais representantes \citeonline{FreemanPryce2009}. A primeira observação é sobre o ciclo que, nessa perspectiva, possui dois níveis de execução (\figref{ciclo-atdd}). Graças a isso, é esperado que a etapa mais externa permanesça por mais tempo em estado de falha.

  \imagem
    {width=6cm}
    {Duplo ciclo da escola londrina de tdd}
    {ciclo-atdd}
    {ciclo-atdd}
    {Próprio autor\footnotemark}
  \footnotetext{Baseado na imagem encontrada em \citeonline[pág. 40]
  {FreemanPryce2009}}

  Também somos forçados a definir um esboço estrutural dos componentes (\figref{esboco-estrutural}) de auto nível sobre o qual repousaremos a solução (inicialmente - pois ela pode, ou não, vir a ser modificado durante o transcorrer do projeto).

  \imagem
    {width=6cm}
    {Esboço estrutural dos componentes da solução}
    {esboco-estrutural}
    {esboco-estrutural}
    {Próprio autor}

  \subsubsection{Criar conta}

    Para levantar as histórias, \citeonline{Cohn2004} sugere o seguinte método: identificar os papéis; utilizá-los para construir narrativas no formato - \narrativa{papel}{fazer alguma coisa}{atingir um objetivo}; levantar os critérios de aceite de cada história.

    Com a história \emph{Criar conta} chegamos a seguinte narrativa: \narrativa{visitante}{criar uma conta}{entrar e jogar uma partida}.

    Também foram levantados os seguintes critérios de aceite: deve ser informado um email válido; deve ser informada uma senha de tamanho mínimo 8 (oito) e tamanho máximo 100 (cem) caracteres; não deve ser possível criar uma conta para um email que já vinculado a uma outra conta; deve constar um registro na base de dados.

    Há elementos para elencarmos os cenários a serem trabalhados (\figref{criar-conta.cenarios}).

    \codigo
      {numbers=none}
      {\textbf{Cenários} elencados para a \textbf{história} \emph{Criar conta}}
      {criar-conta.cenarios}
      {criar-conta.cenarios}
      {Próprio autor}

    Agora escolhemos um para detalhamento e implementação\footnote{\citeonline{Nicieja2018} apresenta um método para derivação de cenários a partir dos critérios de aceite usando a linguagem \emph{Gherkin}, que se baseia no famoso artigo de \citeonline{North2006}.}: \emph{deve ser bem sucedido}. Daí chegamos as passos da \figref{criar-conta.deve-ser-bem-sucedido.cenario}.

    \codigo
      {numbers=none}
      {Passos do \textbf{cenário} \emph{deve ser bem sucedido} da \textbf{história} \emph{Criar conta}}
      {criar-conta.deve-ser-bem-sucedido.cenario}
      {criar-conta.deve-ser-bem-sucedido.cenario}
      {Próprio autor}

  Há de salientar que os passos do cenário descrito são, suficientemente, abstratos para permitir que os não-técnicos o leiam (ou, ainda, desejavelmente, os escrevam) mas não tanto que inviabilizer a sua automatização. Construímos o que \citeonline[pág. 24]{Evans2003} chama de \emph{linguagem ubíqua}. Dentre outros benefícios se pode citar a não fragilidade\footnote{\cite[pág. 93]{RoseWynneHellesoy2015} Brittle features} desses artefatos, tornando-os resilientes às alterações de implementação, enquanto o comportamento esperado se mantiver intácto. Essa característica é fundamental para alicerçar a \emph{refatoração}\footnote{cite{Fowler1999}} e, por sua vez, é condição para o alvo mirado, a \emph{arquitetura limpa}.

  Retomando o esboço estrutural, o enriquecemos com os componentes que substituirão a figura humana no processo, dando vida própria a documentação\footnote{\citeonline[pág. 29]{Adzic2011} Live documentation}, donde se chega a \figref{esboco-estrutural-enriquecido}:

  \imagem
    {width=12cm}
    {Esboço estrutural enriquecido com os componentes que permitem a automatização}
    {esboco-estrutural-enriquecido}
    {esboco-estrutural-enriquecido}
    {Própio autor}

  Nesse ponto usamos o que \citeonline[págs. 142 e 226]{Abelson1996} chama de \emph{wishful thinking}\footnote{Pensamento positivo, em tradução livre.} - mais precisamente o que \citeonline[pág. 45]{Astels2003} denomina \emph{programar por intenção} - para implementar os passos do cenário apresentados como se a funcionalidade pretendida já existisse. O objetivo é ser o mais claro possível sobre o que se almeja com aquela especificação, levando à \figref{criar-conta.deve-ser-bem-sucedido.passos}:

  \codigo
    {numbers=none, basicstyle=\ttfamily\tiny}
    {Implementação dos passos do \textbf{cenário} \emph{deve ser bem sucedido} da \textbf{história} \emph{Criar conta}}
    {criar-conta.deve-ser-bem-sucedido.passos}
    {criar-conta.deve-ser-bem-sucedido.passos}
    {Própio autor}

  Essa etapa pode divergir bastante entre os diversos autores que tratam do tema, no que tange os conceitos de \emph{baby steps}\footnote{\cite[pág. 1]{Aniche2011}}, \emph{iniciar pelas verificações}, \emph{usar somente uma asserção por cenário}, \emph{não usar asserções fora da fase de verificação}. No fim, prevalecem, \emph{a familiaridade com as tecnologias empregadas}, \emph{a experiência no processo de desenvolvimento guiado por testes}, \emph{a simplicidade}\footnote{\cite[pág. TODO]{Beck2003}}, mas, principalmente, \emph{o bom senso}.
