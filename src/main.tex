\documentclass[12pt,a4paper,oneside,english,brazil]{article}
\usepackage[abnt-emphasize=bf,abnt-and-type=e,alf]{abntex2cite}
\usepackage{times}
\usepackage[left=3cm, top=3cm, right=2cm, bottom=2cm]{geometry}
\renewcommand{\baselinestretch}{1.5}
\usepackage{indentfirst}


\usepackage{listings}
\usepackage{courier}

\usepackage{xcolor}
\definecolor{lightgray}{rgb}{0.95,0.95,0.95}
\definecolor{codegray}{rgb}{0.5,0.5,0.5}

\lstdefinestyle{meuestilo}{
  backgroundcolor=\color{lightgray},
	numberstyle=\tiny\color{codegray},
	basicstyle=\ttfamily\footnotesize,
	breakatwhitespace=false,
	breaklines=true,
	captionpos=b,
	keepspaces=true,
	numbers=left,
	numbersep=7pt,
	showspaces=false,
	showstringspaces=false,
	showtabs=false,
	tabsize=2
}

\lstset{style=meuestilo}


\title{
  \textbf{Hive} \\
  \large um estudo de caso do uso de desenvolvimento guiado por testes para
  alcançar uma arquitetura limpa na construção de um jogo para internet com
  múltiplos competidores
}
\author{Hudson Ferreira Leite}
\date{2021}

\begin{document}

% \maketitle

\clearpage

\section{Introdução}
% \label{secIntroducao}
% \normalsize

Desenvolver software sempre foi uma atividadade complexa, que pode, facilmente,
apresentar desafios das mais diveras naturezas: agendamento; má-definição de
requisitos - funcionais ou não; subdimensionamento de recursos;
não-escalabilidade; etc.

Negligenciar esse amálgama de possibilidades é um erro comum, e até esperado,
de profissionais pouco experimentados, mas, não raro, encontram-se exemplares
com bastante quilometragem na carreira.

O Standish Group demostra, em seu \emph{Chaos Report} \cite{ChaosReport2015}, que
quanto maior o projeto maior é a chance de falha. O mesmo relatório, no entanto,
aponta alguns caminhos a seguir para encontrar o sucesso: \emph{uma
arquitetura padrão} e \emph{um processo ágil}.

Ao mencionar agilidade é possível inferir que a capacidade de adaptação é uma
necessidade fundamental para o bom andamento, princípio defendido desde a
primeira hora por este tipo de metodologia.

Quanto à  arquitetura, faz-se evidente que utilizar um padrão é uma decisão que
pode poupar longas, e pouco frutíferas, discursões sobre assuntos já
consolidados.

\section{Testes}

\subsection{Desenvolvimento Guiado por Testes (TDD)?}

Dentre as práticas que sustentam os métodos ágeis, o uso de testes, como uma
disciplina de suporte as demais, é um fato consolidado. Mas a abordagem, também,
conhecida por \emph{testar primeiro} é ainda mais difundida nesse meio.

Não é raro ver profissionais se referindo tal técnica como "desenvolvimento
orientado a testes", e ainda que não se possa dizer que esta expressão está
incorreta, essa é uma consequência da sua aplicação. Mais provavel, seria dizer,
que a tradução mais correta seria "desenvolvimento guiado por testes". A ideia é
extremamente simples \cite[p.1]{FreemanPryce2009}: escrever um teste para um
código ainda inexistente.

Essa postura realoca o papel dos testes, de uma atividade meramente focada em
descobrir defeitos para uma disciplina centrada na experiência do usuário, ao
ajudar os desenvolvedores a entender as reais necessidades daqueles, através de
um processo dialético de construção de conhecimento \footnote{
  \citeonline[p.28]{Oliveira1993} sugere, ao analisar Vygotsky e o processo de
  formação de conceitos, que os últimos são construções culturais,
  internalizadas pelos indivíduos ao longo do processo de desenvolvimento.
}.

Daí porque \citeonline[p.3-5]{FreemanPryce2009} posicionam a prática como
fundamentada em três pilares: 1. aprendizagem; 2. retroalimentação; 3. suporte a
mudança, e de onde se pode concluir se tratar de uma atividade incremental e
iterativa.

\section{Arquitetura de Software}

\subsection{Arquitetura Limpa}

\section{Como tdd auxilia na busca de uma arquitetura limpa?}

\section{O estudo de caso}

\lstinputlisting{App.js}

\section{
  O desenvolvimento dos conceitos citados na implementação do estudo de caso
}

\section{Conclusão}

\clearpage
\renewcommand\refname{Referências Bibliográficas}
\bibliographystyle{abntex2-alf}
\bibliography{refs}

\end{document}
